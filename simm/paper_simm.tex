target: gente de software libre / ciencias de la computación. 

¿Se pueden re-utilizar los chips de la simm? Habiendo una disponibilidad vigente 
de simms en la región.

sim: maximo 16M de ram / 36 contactos  - 8 x 2M c/u 
\---> sencillos de des-soldar/soldar los chips individuales no la simm completa

free rtos (free real time operating system) --> software libre? 
so mas pequeño utilizado en la industria para sistemas que manejen 
mas de un proceso. Tiempo real, muy chiquito, manejo de interrupciones. 

con 2M de RAM sobra para free rtos
linux embebidos con 4m arranca. Con dos chips ya podemos correr un so 
mas potente, uclinux que no mmu. 

por ejemplo una solución que necesite usb, no se si con free rtos podria 
utilizarce. 

Seria recomendable para soluciones que necesiten más ram de lo que trae
el microcontrolador

simm controlada con microcontrolador  (arduino?)
uso real:

Re uso de hardware obsoleto (reciclado) para el desarrollo de open hard actual. 

Background 
Cuál es la situación actual de los desarrolladores de hard embebido en la region

Why
	consumo de energía (green computer)
		\----> 	qué hacen con los componenetes viejos y el uso de 
			energía otros países.
	reciclado de electronica 
	disponibilidad de componenetes en el mercado / costo 
	open hardware 

estre trabajo aims


method used 

results 
	comparamos con otro trabajo 
	medimos performance del micro usando la ram 
	(por ejemplo que el 2% de uso de micro es para referenciar ram)

conclusiones
	cuál es el impacto de controlar la memoria (gráfico)

averiguar maximo y mínimo del cacic



Introduction


Green Computing, green IT. 
http://www.wiley.com/WileyCDA/WileyTitle/productCd-1119970059.html#instructor

Embedded systems: 
New and powerful microcontrollers, good enough fou many custom applications 
like robotics. Which requires less than the functionality provided by a processor (which complexity may be even not a desired feature), but more than basic combinational circuits. 

FreeRTOS 
Take adventage of general purpose operating systems like GNU/Linux, which is highly portable and widely spread (reducing the learning curve to use them). uCos and uclinux.  

http://archive.linuxgizmos.com/ecos-vs-uclinux-which-is-best-for-your-embedded-target-a/


Measurement of environmental impact of each human activity has become part of 
every agenda. Computer science field is not absent in this regard; the terms 
"green computing", or "green IT" has been around since few years ago, gaining
 more and more importance everyday. [San Murugesan]

We came accross re-thinking the recycling processes, not only because of the 
environmental impact, but to address our current (_local_) problematic about
designing new and powerful solutions with the available hardware; that is 
coming from the local market (which is not quite open) and/or using existing 
hardware; reducing costs as much as possible. 


 
Que hicimos

Consultamos con los alumnos del curso de administracion de sistemas si tienen
simm rams disponibles.
El 25% del alumno contesta afirmativamente y provee simm rams.
Se toma uno de los simms y se sueldan N patitas.
Se lo conecta como muestra el proyecto en github, a un microcontrolador (atmega328)
Se programa dos versiones de software (una en C y la otra en assembler)
para controlar la simm y proveer
una API para utilizar la RAM extra disponible, con un simple read/write a
nivel de bytes.
Se realizan tests de ambas versiones.
Se verifica que las simms pueden ser reaprovechadas con muy poco consumo
de recursos del microcontrolador que las controla en los tests (medir cuanta
ram y cpu del micro se utilizan)

