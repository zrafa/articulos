
%%%%%%%%%%%%%%%%%%%%%%% file typeinst.tex %%%%%%%%%%%%%%%%%%%%%%%%%
%
% This is the LaTeX source for the instructions to authors using
% the LaTeX document class 'llncs.cls' for contributions to
% the Lecture Notes in Computer Sciences series.
% http://www.springer.com/lncs       Springer Heidelberg 2006/05/04
%
% It may be used as a template for your own input - copy it
% to a new file with a new name and use it as the basis
% for your article.
%
% NB: the document class 'llncs' has its own and detailed documentation, see
% ftp://ftp.springer.de/data/pubftp/pub/tex/latex/llncs/latex2e/llncsdoc.pdf
%
%%%%%%%%%%%%%%%%%%%%%%%%%%%%%%%%%%%%%%%%%%%%%%%%%%%%%%%%%%%%%%%%%%%


\documentclass[runningheads,a4paper]{llncs}


\usepackage[utf8x]{inputenc}							
\usepackage{amsmath}
\usepackage{graphicx}
\usepackage{xcolor}
\usepackage{listings}
\usepackage{fix-cm}

\usepackage{multirow}

\usepackage[spanish,es-tabla]{babel}							
\usepackage{amssymb}
\setcounter{tocdepth}{3}
\usepackage{graphicx}

\usepackage{fancyvrb}
\newcommand\userinput[1]{\textbf{#1}}



\usepackage{url}
\urldef{\mailsa}\path|{rafa, mtl, rdc}@fi.uncoma.edu.ar| 
%\urldef{\mailsb}\path|anna.kramer, leonie.kunz, christine.reiss, nicole.sator,|
%\urldef{\mailsc}\path|erika.siebert-cole, peter.strasser, lncs}@fi.uncoma.edu.ar| 


\newcommand{\keywords}[1]{\par\addvspace\baselineskip
\noindent\keywordname\enspace\ignorespaces#1}

\begin{document}

\mainmatter  % start of an individual contribution

% first the title is needed
\title{Arquitectura e Implementación de Equipamiento Prototipo de Bajo Costo para la Medición del Diámetro de Fibras de Cabra en Campo}

% a short form should be given in case it is too long for the running head
\titlerunning{Arquitectura de Prototipo para la Medición del Diámetro de Fibras} 

% the name(s) of the author(s) follow(s) next
%
% NB: Chinese authors should write their first names(s) in front of
% their surnames. This ensures that the names appear correctly in
% the running heads and the author index.
%
\author{Rafael Zurita
% \thanks{Please note that the LNCS Editorial assumes that all authors have used
%the western naming convention, with given names preceding surnames. This determines
%the structure of the names in the running heads and the author index.}%
\and Miriam Lechner\and Rodolfo del Castillo\and Eduardo Aisen}
%\and Miriam Lechner\and Rodolfo del Castilo\and Eduardo Aisen\and\\
%Anna Kramer\and Leonie Kunz\and Christine Rei\ss\and\\
%Nicole Sator\and Erika Siebert-Cole\and Peter Stra\ss er}
%
%\authorrunning{Lecture Notes in Computer Science: Authors' Instructions}
\authorrunning{Arquitectura de Prototipo para la Medición del Diámetro de Fibras} 
% (feature abused for this document to repeat the title also on left hand pages)

% the affiliations are given next; don't give your e-mail address
% unless you accept that it will be published
\institute{Facultad de Informática y Facultad de Agronomía,\\ Universidad Nacional del Comahue\\
Buenos Aires 1400, Neuquén, Argentina\\
\mailsa\\
%\mailsb\\
%\mailsc\\
\url{http://faiweb.uncoma.edu.ar}}

%
% NB: a more complex sample for affiliations and the mapping to the
% corresponding authors can be found in the file "llncs.dem"
% (search for the string "\mainmatter" where a contribution starts).
% "llncs.dem" accompanies the document class "llncs.cls".
%

\toctitle{Lecture Notes in Computer Science}
\tocauthor{Authors' Instructions}
\maketitle


\begin{abstract}
The abstract should summarize the contents of the paper and should
contain at least 70 and at most 150 words. It should be written using the
\emph{abstract} environment.
\keywords{We would like to encourage you to list your keywords within
the abstract section}
\end{abstract}


\section{Introducción}

Los pequeños productores en el sistema agroecológico del norte de la Provincia de Neuquén son los crianceros. A nivel local la palabra criancero denomina a un amplio conjunto de productores familiares en el que predominan productores campesinos o con rasgos campesinos y que se dedican fundamentalmente a la cría de caprinos y ovinos (Bendini y col., 2002). Viven de la producción de fibras textiles de origen animal (principalmente caprino) en el norte neuquino, y no disponen de los medios técnicos y el asesoramiento profesional requerido para: i) orientar la selección animal a la producción de vellones mas pesados y calidades de fibra de valor superior; ii) establecer objetivamente los parámetros de la calidad de la fibra, de manera de poder clasificarla para optimizar la comercialización; iii) negociar con el comprador el precio de mercado que la calidad de la fibra amerita; y iv) orientar la producción ganadera hacia aquellos biotipos mejor adaptados a la naturaleza de la región, para así poder frenar y eventualmente revertir el proceso de desertificación que las afecta.

Sin los medios técnicos la cosecha de pelo de cabra criolla mediante esquila se realiza con una selección visual subjetiva del tipo de animal, llegando a descartar aquellos ejemplares con fuerte componente de vellón tipo lustre (angorizado), y separando los lotes en blanco y color. Este producto que ingresa a destino (planta textil) requiere de una nueva separación y clasificación antes de entrar al proceso de descerdado (que en muchos casos resulta impracticable dado que los vellones no pueden identificarse y sólo se obtienen “parches” sin relación entre sí).

Por otro lado, existen en el mercado varios equipos profesionales para la determinación de finura (desarrollados principalmente para lana) como el OFDA 2000, WoolView 20/20, LaserScan, air flow, y lanámetro, entre otros. De cualquier manera, la mayoría de estos equipamientos son costosos y destinados a ambientes de laboratorio, por lo que no son indicados para utilización sobre el animal. Woolview 20/20 fue diseñado para trabajo en campo, pero el producto está discontinuado.
En este artículo se presenta la arquitectura de hardware y software de un equipamiento prototipo desarrollado en la Universidad Nacional del Comahue (por lo que se lo ha denominado ``prototipo UNCOMA'') para la medición de diámetro de fibra sobre el animal en campo. Considerando al criancero como destinatario del 
producto final, y no un entorno de laboratorio. En tal sentido, se persiguen al menos dos objetivos que 
resulten en la factibilidad de uso: entrenamiento previo mínimo y, preparación sencilla de la muestra a 
analizar.  
 
Los resultados muestran que el costo, performance y precisión del prototipo UNCOMA propuesto, es adecuado para la selección objetiva en origen de animales de alta producción de fibra (especialmente cashmere), lo que permitirá aumentar la producción por cabeza e incrementar significativamente los ingresos anuales del productor campesino. De este modo se podrán efectivizar las propuestas de manejo de la carga animal, contribuyendo así a revertir el grave problema de la desertificación.

El resto de este trabajo se estructura de la siguiente manera. En la sección 2 se describen
las arquitecturas de software y hardware del equipamiento propuesto, con un énfasis en
el método de medición por análisis de imagen digital. En la sección 3 se presentan varios
resultados experimentales, para validar el proceso llevado a cabo por el prototipo.
Por último, en la sección 4, se presentan las conclusiones y líneas de trabajo a futuro.

\section{Arquitectura}

\subsection{Arquitectura de Hardware}

La arquitectura del hardware presenta dos componentes principales :
\begin{itemize}
\item El dispositivo de mano portátil;
\item El hardware de procesamiento embebido
\end{itemize}

El dispositivo de mano portátil contiene un microscopio digital, un led de contraste, una prensa, y un mecanismo de gatillo.
El microscopio es de 400 ampliaciones, modificado y calibrado. Funciona con una tensión de 5 voltios, y mecánicamente se encuentra empotrado en el dispositivo, como se observa en Figura~\ref{fig:prototipo}.
\begin{figure}
\centering
\includegraphics[height=6.2cm]{prototipo}
\caption{Equipamiento prototipo UNCOMA}
\label{fig:prototipo}
\end{figure}

El mecanismo de gatillo realiza varias tareas en un único paso. Al gatillar, el operador prensa el acrílico opuesto a la óptica del microscopio, lo que produce que las fibras sean enfocadas correctamente y queden firmes. Además, al final del accionar del gatillo, se activa un interruptor digital. Este interruptor envía una señal al hardware embebido, a través de un puerto de E/S digital que dispara un evento en el sistema y activa el software de obtención de imágenes.
En ese momento, el software en el hardware embebido solicita una captura de imagen digital de las fibras que el microscopio tiene enfocadas y estancadas, para luego ser procesada. La imagen digital es obtenida desde el dispositivo portátil a través de una interfaz USB 2.0, la cual interconecta ambos componentes (el dispositivo de mano y el hardware embebido).

El hardware embebido es el sistema 
que analiza la imagen digital y reporta los resultados.
Un diagrama de bloques del hardware embebido es mostrado en la Figura 1.2



Block Diagram
Features
CPU ARM Cortex™-A8 1.5GHz I
Memory 512MB de 32-bit DDR3 RAM
4GB NAND FLASH
WiFi 802.11b/g/n network card. 
4 SDIO interfaces (SD 3.0, UHI class)
USB 2.0
GPIO

FOTO DEL MICROSCOPIO Y EL HARDWARE EMBEBIDO

La facilidad de manejo y peso del dispositivo portátil de mano posibilita que las muestras de fibras a ser procesadas no deban ser preparadas de antemano. El operador puede tomar capturas directamente sobre el animal en campo.


\subsection{Arquitectura de Software}


La arquitectura de software puede observarse en la Figura 2.
A bajo nivel, el hardware embebido está controlado por el sistema operativo Debian. Particularmente, los principales drivers que se utilizan del kernel Linux son el universal video class (UVC) y la interfaz de E/S de puertos generales (GPIO) del núcleo.
A través de uno de los puertos de entrada digitales (GPIO) se recibe la señal digital disparada por el dispositivo portátil. En este punto, la aplicación en espacio de usuario recibe una notificación del evento y realiza el proceso de captura y análisis digital de una imagen de fibras. El proceso general puede ser observado en la Figura~\ref{fig:proceso}.
\begin{figure}
\centering
\includegraphics[height=0.9cm]{proceso}
\caption{Etapas del proceso llevada a cabo por el prototipo UNCOMA para la captura y análisis de la imagen digital de fibras.
}
\label{fig:proceso}
\end{figure}


Para la captura de la imagen, la aplicación embebida utiliza las facilidades que el kernel Linux exporta al espacio de usuario a través del driver UVC. El driver UVC que controla la captura del microscopio incorporado en el dispositivo portátil obtiene una imagen completa de las fibras en la prensa, y se la presenta a la aplicación en espacio de usuario.
El formato de la imagen digital recibida es JPEG, por lo que se procede, en primera instancia, a un proceso de conversión al formato de mapa de grises portable (PGM). Esto es llevado a cabo por el conjunto de herramientas de software netpbm, y es necesario debido a que la aplicación embebida requiere este formato como entrada del algoritmo {\it line segment detector (LSD) [REF]}.
La resolución de la imagen es de 640x480 píxeles, en donde 1 píxel corresponde a 0.9743 micrones de acuerdo a la óptica y microscopio seleccionados para el prototipo.

Cada fibra contiene una médula, eventualmente la imagen capturada puede hacerla visible lo cual afecta a 
las mediciones. Por tal motivo se ejecuta un filtro de médula para quitarlas parcialmente. El filtro se realiza calculando la transformada de distancia para píxeles que se encuentren dentro de un rango de grises. 
Dicho rango fue previamente establecido y representa los valores esperables para la médula. Si la detección es positiva, se reemplaza la zona medular por píxeles que contengan un color dentro del rango de grises  
coincidente con fibra, que también fue previamente establecido.

Posteriormente, la aplicación utiliza el algoritmo {\it line segment detector (LSD) [REFERENCIA]} para obtener un conjunto de segmentos de línea que representan a la imagen original. A partir de este conjunto se 
realiza la medición de diámetros de fibras según el algoritmo a continuación:

\begin{Verbatim}[commandchars=\\\{\}]
    #
    # Pseudocódigo del algoritmo de cálculo de diámetro medio de fibras
    # ENTRADA: Conjunto de segmentos de línea (lsd_set)
    # SALIDA: Mediciones y diámetro medio (mediciones_set, diametro)
    #
    mediciones = 0;
    diametro = 0;
[1] FOR each segmento IN lsd_set DO

      diametro_tentativo = 0;
      mediciones_t = 0;
[2]   FOR each punto (x,y) IN inicio_medio_final(segmento) DO

        f() = perpendicular(segmento, punto);
[3]     FOR each seg2 IN lsd_set DO

          IF paralelos(segmento, seg2) AND 
             ( f() == perpendicular(seg2) ) AND
             ( f(DESDE segmento HASTA seg2) == pixels_de_fibra() ) THEN

[4]             diametro_tentativo = distancia( f(DESDE segmento HASTA seg2) );
                mediciones_t++;
          END IF
        END FOR
      END FOR

[5]   IF (mediciones_t >= 2) THEN 
        diametro = diametro + diametro_tentativo;
        mediciones++;
        mediciones_set = add(diametro_tentativo);
      END FI
    END FOR

    diametro = diametro / mediciones;

    RETURN mediciones_set, diametro
\end{Verbatim}

 
De manera general, el algoritmo busca segmentos paralelos intersectados por dos a tres perpendiculares.
Interpretando esto como un segmento de fibra. Si los píxeles contenidos entre los segmentos, a lo largo de
 las  perpendiculares, coinciden con el rango de color esperable para fibra, entonces se consideran dichas
distancias como diámetros tentativos. Calculando por último el promedio de los valores obtenidos para 
cada segmento. Específicamente:   


\begin{enumerate}
\item Por cada segmento de línea en el conjunto se calcula la ubicación de 3 puntos (x,y):
principio medio y final. 
\item Por cada punto se calcula la función de la recta perpendicular al segmento original.
\item Se recorren todos los segmentos restantes en búsqueda de aquellos que sean paralelos al segmento original, e intersectados a su vez por la perpendicular calculada en el paso anterior.
Si el segmento perpendicular desde el punto origen hasta la intersección contiene puntos que no pertenecen al rango de color esperable para fibras, se descarta.
\item Si no se descarta,  el segmento de línea paralelo encontrado es el borde opuesto al original de la fibra, y se procede a realizar el cálculo de distancia entre el punto original y la intersección. Esta medida es catalogada como diámetro tentativo.
\item Si al menos dos puntos obtuvieron un diámetro tentativo se re-cataloga a la medición tentativa como válida, y pasa a formar parte del cálculo estadístico de media de diámetro general. Sino, el diámetro tentativo es catalogado como ruido y se descarta.
\end{enumerate}

	Al final del procesamiento de segmentos de línea se realiza el cálculo de media de diámetro general, la desviación estándar y la varianza.
	Los resultados son emitidos por la salida estándar, y de acuerdo a los valores, se muestra en el display LCD, del hardware embebido, la siguiente información :
\begin{Verbatim}
[1] diámetro medio;
[2] porcentaje de medidas menores a 17um;
[3] porcentaje de medidas entre 17um y 30um;
[4] porcentaje de medidas mayores a 30um;
\end{Verbatim}

Toda la estadística es almacenada en el dispositivo, y puede ser extraída y visualizada a través de su conexión wireless. Se ha desarrollado una interfaz gráfica (GUI) para visualizar el proceso de análisis en una PC, y corroborar fehacientemente (visualmente) que el análisis automático es correcto (que las mediciones realizada por el software se encuentren dentro de las fibras, y sean perpendiculares a sus extremos). Esta interfaz gráfica permite también utilizar el dispositivo portátil (su microscopio) desde una notebook o PC. Un ejemplo de análisis utilizando la interfaz gráfica se muestra en la Figura~\ref{fig:captura2}.
\begin{figure}
\centering
\includegraphics[height=7cm]{captura2}
\caption{Resultado visual de medición de diámetro presentado por la interfaz software. A la izquierda se observa la imagen original obtenida desde el microscopio. A la derecha una representación del resultado del algoritmo de medición. Los segmentos en color rojo son mediciones finales obtenidas por el algoritmo. Esta representación visual es muy útil para corroborar manualmente si hubieron mediciones erróneas del software (por ejemplo, fuera de las fibras).
}
\label{fig:captura2}
\end{figure}




En estudios anteriores se ha establecido que el entrecruzamiento de fibras distorsiona los resultados, por lo que se han propuesto diferentes algoritmos para la detección de los mismos.
El trabajo propuesto en este artículo no sufre de la distorsión de los resultados por el entrecruzamiento de fibras. 
La representación a través de un conjunto de segmento de lineas y el proceso de localización de paralelas opuestas en cada fibra no necesita de una detección suplementaria de entrecruzamiento. Los resultados expuestos en las secciones siguientes, junto a las verificaciones realizadas en campo, muestran también que no se necesita preparación previa de las fibras (las mismas pueden estar en cualquier posición y entrecruzadas).



	

\subsubsection{Algoritmo e implementación para la medición del diámetro de fibra a través de análisis digital de imágenes}

Acá explicamos lo que hace nuestro prototipo :
Toma una foto a través del driver Linux UVC
La convierte a PGM
La analiza con LSD modificado:
1. Quitamos la médula
2. LSD
3. Calculamos los diámetros :
a. Como el paper del chino Shien Li

3. Implementación?
Podemos dar detalles de que está escrito en ANSI C y es portable. Etc? O colocamos lo de implementac               x xión en Resultados?

\section{Resultados}

Para determinar la validez del prototipo se han realizado mediciones de tiempo y precisión.
El orden del tiempo de ejecución del prototipo es importante porque nos permite evaluar si el hardware y software es adecuado para uso cotidiano. Un dispositivo que demore demasiado en obtener los resultados puede no ser indicado para un campo con mucho ganado. En caso de que el equipamiento demore demasiado en cada medición se requeriría un mayor numero de operarios y dispositivos, para realizar mediciones en paralelo; lo cual sería mas costoso.
La precisión del resultado del análisis es la valuación mas importante, ya que un dispositivo que realiza mediciones incorrectas o muy poco precisas no es confiable.


\subsection{Orden del Tiempo de Ejecución}
En el gráfico X se observa el costo en tiempo de ejecución al analizar 96 muestras de fibras reales (sin contar el tiempo del operario en preparar las fibras a medir sobre el animal). El tiempo ha sido observado mediante la herramienta GNU time, el cual ejecuta cada etapa y evalúa los recursos utilizados. Se observa que tanto el proceso de obtención de la imagen, como la conversión de la misma es constante, aunque no así el análisis de la imagen mediante el software propuesto, el cual varía en el rango de (0seg..3seg].
Durante el análisis de la imagen el mayor costo en tiempo de ejecución es el del algoritmo LSD, el cual es de orden de tiempo lineal [REFERENCIA]. Debido a que las fotos son siempre de la misma resolución y el número de de fibras analizadas en cada medición es menor a 15 (un mayor numero de fibras distorsionaría los resultados) el tiempo de ejecución de cada medición ha sido determinado, luego de analizar 96 muestras, en un rango (0seg..3seg].

\subsection{Precisión del proceso de análisis}

Para la obtención de resultados de precisión se analizan dos clases de mediciones realizadas: 
\begin{itemize}
\item mediciones con imágenes de fibras artificiales que contienen medidas conocidas a priori, y
\item mediciones con 96 imágenes de muestras de fibras de un mismo mechón, utilizando el prototipo propuesto en este trabajo y el producto comercial WoolView 20/20. Las muestras se obtuvieron en ambos equipamientos independientemente.
\end{itemize}

\subsubsection{Medición de imágenes artificiales}

Se han confeccionado 74 imágenes de fibras artificiales mediante el software de manipulación de imágenes GIMP. Las imágenes contienen entre 1 y 7 lineas negras (fibras artificiales) de orientación  y sentidos aleatorios. Cada línea tiene un grosor conocido a priori y documentado en toda su extensión. Por ejemplo, en la figura X se observa una imagen con 5 fibras artificiales, cada una de un grosor definido en toda su extensión. Los diámetros seleccionados para las líneas se encuentran entre los 10 píxeles de grosor y los 40 píxeles. La media observada manualmente para cada imagen es calculada como la sumatoria de las medidas de cada línea en la imagen / la cantidad de líneas.
En la figura X2 se observa la medición realizada automáticamente para todas las imágenes por el prototipo, y también la media calculada manualmente. 

\subsubsection{Resultados del método de medición de diámetro de fibra} 

Para el análisis del software propuesto se realizó una comparativa de medición entre el prototipo UNCOMA y el equipamiento Woolview 20/20. 
Woolview 20/20 es un dispositivo comercial para la medición de fibras de lana con precisión menor a un micrón, el cual puede analizar las muestras sin quitar las fibras del animal. 

Con el prototipo UNCOMA se analizaron 96 muestras obtenidas de un mismo mechón de animal. Estas muestras fueron almacenadas para ser publicadas en conjunto con este trabajo. Se desarrolló un script por lotes que secuencialmente analiza cada una de estas muestras almacenadas, utilizando el software prototipo. La estadística de cada foto digital analizada es almacenada en un archivo separado. Tanto las muestras originales como el resultado de cada medición de diámetro a través del software del prototipo puede obtenerse de [3].

La medición utilizando el equipamiento Woolview 20/20 fue realizada por un operario diferente al que realizó la medición con el prototipo UNCOMA. Utilizando el mismo mechón de animal se realizaron 96 mediciones con Woolview 20/20. Este equipamiento acumula la estadística, por lo que se obtuvieron los detalles de las estadísticas intermedias cada 10 mediciones. 

En la Figura~\ref{fig:prototipovswv} se observan los resultados de las mediciones intermedias de ambos equipamientos.
El diámetro medio final obtenido por el prototipo propuesto es de 20,6263 micrones (en la Figura~\ref{fig:histograma} se observa el histograma normalizado de todas las mediciones). El diámetro medio obtenido con el equipamiento Woolview 20/20 es de 20,9 micrones. En la tabla (\ref{tabla:sencilla}) se observan las estadísticas finales de ambos equipamientos.

\begin{figure}
\centering
\includegraphics[height=6.2cm]{prototipovswv}
\caption{Comparativa acumulativa de 96 mediciones utilizando dos equipamientos : prototipo UNCOMA y WoolView 20/20.
La diferencia de la media de diámetro entre ambos equipos va disminuyendo a medida que se realizan mayor cantidad de mediciones sobre el mismo mechón de fibras. Luego de unas 50 muestras la diferencia entre ambos es menor a 0.5 micrones.}
\label{fig:prototipovswv}
\end{figure}

\begin{figure}
\centering
\includegraphics[height=6.2cm]{histograma}
\caption{Histograma normalizado de las 96 mediciones realizadas con el prototipo UNCOMA.}
\label{fig:histograma}
\end{figure}

\begin{table}[htbp]
\begin{center}
\begin{tabular}{|l|l|l|}
\hline
& prototipo UNCOMA & WoolView 20/20 \\
\hline \hline \hline
Nro. de imágenes analizadas & 96 & 96 \\ \hline
Diámetro Medio & 20,6263 & 20,9 \\ \hline
Coeficiente de confort & 91,6\% & 91\% \\ \hline
Desvío estándar & 4,03um & 5,5um \\ \hline
Coeficiente de variación & 19,5\% & 27\% \\ \hline
Muestras dentro del DE & 82\% & No reportado \\ \hline
\end{tabular}
\caption{Estadística global de ambos equipamientos luego de finalizar las mediciones.}
\label{tabla:sencilla}
\end{center}
\end{table}


\section{Conclusiones y discusión}
Siendo la preparación de las fibras sobre el animal o en laboratorio un proceso que se encuentra en el orden de los minutos (, el tiempo observado en el análisis mediante el prototipo no es de un costo significativo, y ha sido aceptado  en las mediciones de prueba realizadas 

\section{Trabajo Futuro}

Referencias



\subsection{Formulas}

\begin{equation}
  \psi (u) = \int_{o}^{T} \left[\frac{1}{2}
  \left(\Lambda_{o}^{-1} u,u\right) + N^{\ast} (-u)\right] dt \;  .
\end{equation}



\subsection{Program Code}

Program listings or program commands in the text are normally set in
typewriter font, e.g., CMTT10 or Courier.


\subsection{Citations}

For citations in the text please use
square brackets and consecutive numbers: \cite{jour}, \cite{lncschap},
\cite{proceeding1} -- provided automatically
by \LaTeX 's \verb|\cite| \dots\verb|\bibitem| mechanism.

\section{BibTeX Entries}

The correct BibTeX entries for the Lecture Notes in Computer Science
volumes can be found at the following Website shortly after the
publication of the book:
\url{http://www.informatik.uni-trier.de/~ley/db/journals/lncs.html}

\subsubsection*{Acknowledgments.} The heading should be treated as a
subsubsection heading and should not be assigned a number.

\section{The References Section}\label{references}

{\it Example of a Computer Program}
\begin{verbatim}
program Inflation (Output)
  {Assuming annual inflation rates of 7%, 8%, and 10%,...
   years};
   const
     MaxYears = 10;
   var
     Year: 0..MaxYears;
     Factor1, Factor2, Factor3: Real;
   begin
     Year := 0;
     Factor1 := 1.0; Factor2 := 1.0; Factor3 := 1.0;
     WriteLn('Year  7% 8% 10%'); WriteLn;
     repeat
       Year := Year + 1;
       Factor1 := Factor1 * 1.07;
       Factor2 := Factor2 * 1.08;
       Factor3 := Factor3 * 1.10;
       WriteLn(Year:5,Factor1:7:3,Factor2:7:3,Factor3:7:3)
     until Year = MaxYears
end.
\end{verbatim}
%
\noindent
{\small (Example from Jensen K., Wirth N. (1991) Pascal user manual and
report. Springer, New York)}


In order to permit cross referencing within LNCS-Online, and eventually
between different publishers and their online databases, LNCS will,
from now on, be standardizing the format of the references. This new
feature will increase the visibility of publications and facilitate
academic research considerably. Please base your references on the
examples below. References that don't adhere to this style will be
reformatted by Springer. You should therefore check your references
thoroughly when you receive the final pdf of your paper.
The reference section must be complete. You may not omit references.
Instructions as to where to find a fuller version of the references are
not permissible.

We only accept references written using the latin alphabet. If the title
of the book you are referring to is in Russian or Chinese, then please write
(in Russian) or (in Chinese) at the end of the transcript or translation
of the title.

The following section shows a sample reference list with entries for
journal articles \cite{jour}, an LNCS chapter \cite{lncschap}, a book
\cite{book}, proceedings without editors \cite{proceeding1} and
\cite{proceeding2}, as well as a URL \cite{url}.
Please note that proceedings published in LNCS are not cited with their
full titles, but with their acronyms!

\begin{thebibliography}{4}

\bibitem{jour} Smith, T.F., Waterman, M.S.: Identification of Common Molecular
Subsequences. J. Mol. Biol. 147, 195--197 (1981)

\bibitem{lncschap} May, P., Ehrlich, H.C., Steinke, T.: ZIB Structure Prediction Pipeline:
Composing a Complex Biological Workflow through Web Services. In: Nagel,
W.E., Walter, W.V., Lehner, W. (eds.) Euro-Par 2006. LNCS, vol. 4128,
pp. 1148--1158. Springer, Heidelberg (2006)

\bibitem{book} Foster, I., Kesselman, C.: The Grid: Blueprint for a New Computing
Infrastructure. Morgan Kaufmann, San Francisco (1999)

\bibitem{proceeding1} Czajkowski, K., Fitzgerald, S., Foster, I., Kesselman, C.: Grid
Information Services for Distributed Resource Sharing. In: 10th IEEE
International Symposium on High Performance Distributed Computing, pp.
181--184. IEEE Press, New York (2001)

\bibitem{proceeding2} Foster, I., Kesselman, C., Nick, J., Tuecke, S.: The Physiology of the
Grid: an Open Grid Services Architecture for Distributed Systems
Integration. Technical report, Global Grid Forum (2002)

\bibitem{url} National Center for Biotechnology Information, \url{http://www.ncbi.nlm.nih.gov}

\end{thebibliography}


\section*{Appendix: Springer-Author Discount}

order to obtain the discount.

\section{Checklist of Items to be Sent to Volume Editors}
Here is a checklist of everything the volume editor requires from you:


\begin{itemize}
\settowidth{\leftmargin}{{\Large$\square$}}\advance\leftmargin\labelsep
\itemsep8pt\relax
\renewcommand\labelitemi{{\lower1.5pt\hbox{\Large$\square$}}}

\item The final \LaTeX{} source files
\item A final PDF file
\item A copyright form, signed by one author on behalf of all of the
authors of the paper.
\item A readme giving the name and email address of the
corresponding author.
\end{itemize}
\end{document}
